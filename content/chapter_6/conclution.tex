\chapter{Conclusion}\label{chap:Conclu}

\minitoc


%\section{The origin}
Finally it is the end, I hope that was easer and faster to read it then for me to write it.

During this thesis, coverage path planning problem is solved with new adaptive methods. The new methods introduced have been developed based on all the work carried out previously. 
About the coverage path planning a naive idea is to apply a sweeping inside the area to cover. This method can be efficient in numerous cases but may have many disadvantages as, the size of the path (usually longer) 
or the low robustness of this method in the to additional constraints (as a path in 3 dimension, complex area, trajectory constraint,...). In fact,  the different methods based on sweeping are workable but not much optimised and rigid to numerous constraints. \\
The solution proposed is a two steps optimisation by splitting the problem of coverage path planning into two sub-problems. 
The first sub-problem is to find the best waypoints position in order to cover the area. 
%The goal of this first sub-problem is to estimate the best position for a set of waypoints. 
The best position for a set of waypoints can be lead by several indicators and constraints depending on the application, but in our case the main interest is the coverage rate of the area. Based on it, a cost function has been developed to estimate the coverage rate and integrate some of the predefined constraints.\\
%The best position depends then few constraints and the more important aspect to evaluate the waypoints position is the global coverage of the areas. 
Consequently the goal became to find the best way to optimise the position for a set of waypoints, in order to maximize the coverage depending the additional constraints. The problem of positioning waypoints is tricky and cannot be solved perfectly at each time (np-hard problem), hence the need to find an acceptable approximation to position the waypoints. An acceptable solution is the solution which maximizes the coverage and minimizes the potential constraints, so that even if the solution found is not perfect it remains exploitable and can be considered as the best possible.
The set of waypoints positioning is closely related to the camera positioning. We saw in the state of the art the principal solutions applied to optimize the problem of camera positioning.\\
Among them the meta-heuristic with the evolutionary algorithm family and in particular the PSO (Particular Swarm Optimisation) are the most commonly used and most promising.
 The PSO may offer a fast solution with is flexible and easily tunable arguments for different constraints. Despite the great potential of PSO, another algorithm is introduced and compared to provide fine results for our problems.
The GA (Genetic Algorithms) from the same family than PSO, has been under estimated for this problem. During this thesis, the GA have been investigated more carefully. 
To distinguish between the 2 algorithms, the GA and PSO have been compared in several condition to can conclude on the advantage of using GA, especially for the camera positioning in the wide and complex area. When the PSO is faster and close to the optimal solution in the small area.
Finally the solution proposed is to combine the advantage of these two algorithms to offer an appropriate and flexible solution by using a hybridized GAPSO. 

The GAPSO implies a longer computation, but provide significant advantages such as better efficiency and also more flexibility to the area size and number of  cameras to optimize.
%The GAPSO used has the advantage to offer a flexible and appropriate answer to all the situation but that implies a longueur computation. 
The innovative solution is introduced to upgrade optimization of the waypoints positioning problem thus allow the first sub-problems to have a better solution.

The second sub-problem is to find a path passing by all the waypoints positioned using the GAPSO. The problem of estimating the shorter path passing by a set of given waypoints is closely related to a well known problem. The TSP (Traveller Sell-man Problem) has the same goal and numerous works have been done to find the best algorithms to solve it. Unforgettably, the TSP is one of the NP-complete problem and cannot be completely solved. The best way to get an acceptable solution for the TSP is to use a GA for combinatory problem as advocated in the literature. 
The idea is to find the good order to pass by all the waypoints before to come back to the starting one, while  minimizing the length path. 
In this thesis this two sub-problems have been addressed and an efficient answer has been proposed. 

Finally  our main contribution is based on an original problem splitting to offer 2 independent optimization phases. Where some advanced algorithm has been applied to solve more efficiently the problem of waypoints (or cameras) positioning, an important work has been made on the algorithms for the waypoints positioning. This allows us to affirm that an appropriate GA is more efficient than the PSO for large spaces and for a large number of waypoints despite previous works in the literature. Moreover the GAPSO allows an even more robust solution to constraints and give more flexibility to the proposed solution. 

The proposed solution can design an efficient path planning are at same time, shorter and more adaptable to the constraints. During the experiments presented few constraints have been added and the solution proposed shown its adaptability and efficiency. 
%The advantage of the solution proposed is the adaptability and flexibility to new constraint.  A future work  may be focused on the adding new constraints. 
The advantage of the solution proposed is the adaptability and flexibility to new constraint and  a future work may be based on it. %Among the constraint unexplored until now . In fact, more work can be done to estimating an efficient path planing depending then constraint as for example mobility constraint.  
Among the constraints unexplored so far, the one concerning the creation of the path is probably the most interesting.  Indeed many constraints of trajectory can be added to obtain an optimized path for different kind of UAVs. This type of constraints must be added according to physical property of the UAV used and deserve to be experimentally validated. 
The experimental validation is also an important track of research. In fact, in this thesis the experimentation proposed are only based on simulation and the addition of robotic experimentations may be a good way to validate and more over add some environmental constraints.  
An other perspective for a future work can be to use the waypoints positions and path planning computed as an initial guard tour before to work on a dynamic adaptation depending on the environment.



... Correct and continue ...

 
% The smart idea is to decompose the problem the coverage path planning problem into two problems. 
% from CPP : 
% The conclusion is despite the low quality of the answer obtained with these methodologies the interesting track and possible gain must push the research to continue to explore the GAPSO optimization to solve the problem of coverage path planning problem as a one optimization problem. But caution the increased search space can become break of the optimization despite the reducing of the number of dimensions to optimize. In fact, the number of dimensions is reduced by fixing the altitude and the camera orientation and moreover with this unique optimization the number of required waypoints are greatly reduced. 
%The main risk of using a unique optimization for the problems of CPP is the complexity and the associate search space. In fact, among the reason to explain the inconclusive results, the complexity due to the fusion of 2 NP-hard and NP-complete problems could be the cause. \\
%		
		
		Sum up,\\ 
		advantage theses\\
		future resarch